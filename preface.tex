% !TeX spellcheck = th_TH

%%%%%%%%%%%%%%%%%%%%%%
\usepackage[dvipsnames]{xcolor}
\usepackage[framemethod=TikZ]{mdframed}

\usepackage{fancyhdr}
\usepackage{multicol, array, tabularx, enumitem}
%%%%%%%%%%%%%%%%%%%%%%


% Set the locale for linebreak to Thai
\XeTeXlinebreaklocale "th"
% In English, when TeX tries to justify text,
% it will add some spaces between words
% For Thai, we "must not" add any space between words
% i.e. put "zero" space between words
\XeTeXlinebreakskip = 0pt plus 0pt
% For a bit better(?) justified output
\sloppy

\usepackage[top = 1in, bottom = 1in, left=1in, right = 1in]{geometry}

\usepackage{amsthm, amsmath, amssymb, mathtools, indentfirst}


%\usepackage[sb]{libertinus}
\usepackage{lmodern}

% For any unicode characters, require XeTeX/XeLaTeX
\usepackage{fontspec}
\defaultfontfeatures{Mapping=tex-text} 
\setmainfont{Latin Modern Roman}
% Set main fonts
% For Thai, I recommend to scale the size to the uppercase size of latin alphabet
%\setmainfont[Scale=MatchLowercase,Mapping=tex-text]{TH Sarabun New}
%%\setmainfont{TeX Gyre Termes}				% Free Times
%%
%%% Sans font
%%\setsansfont{TeX Gyre Heros}				% Free Helvetica
%%
%%% Monospace font
%%\setmonofont{TeX Gyre Cursor}				% Free Courier
%\SetSymbolFont{operators}{normal}{OT1}{tex-gyre-pagella}{m}{n}
%\DeclareSymbolFont{symbols}{OMS}{cmsy}{m}{n}
%\DeclareSymbolFont{largesymbols}{OMX}{cmex}{m}{n}

%\setmathfont{LibertinusMath-Regular.otf}

%% Because latin font in Sarabun is Sans Serif, we prefer to use Serif font
\newfontfamily{\thaifont}[Scale=MatchLowercase,Mapping=tex-text]{TH Sarabun New}

% Set environment for Thai fonts
\newenvironment{thailang}
{\thaifont}
{}

% For automatic switching between languages
\usepackage[Latin,Thai]{ucharclasses}

% When using Thai characters use thailang environment
\setTransitionTo{Thai}{\begin{thailang}}
% For other characters, switch back to the original environment
\setTransitionFrom{Thai}{\end{thailang}}

% Single spacing is too tight for Thai
\usepackage[nodisplayskipstretch]{setspace}
\onehalfspacing

\usepackage{etoolbox}

% For thaialph numbering \thAlph
\usepackage{polyglossia}          
% Set the normal language to English
% i.e. numbering, latin characters will use English font
\setdefaultlanguage{english}
% When using Thai characters, the font will be automatically changed to Thai font
\setotherlanguage{thai}

\AtBeginDocument\captionsthai               % Force the caption to Thai

%%%%%%%%%%%%%%%%%%%%%%%%%%%%%%%%%%%%%%%%%%%%%%%%%%%%%%%%%%%%%%%%%%%%%%%%%%

\theoremstyle{definition}
\newtheorem{problem}{ปัญหาข้อที่}
\newtheorem{definition}{บทนิยาม}[chapter]
\newtheorem{example}{ตัวอย่าง}[section]
\newtheorem{theorem}{ทฤษฎีบท}[chapter]
\newtheorem{lemma}{บทตั้ง}[section]
\newtheorem{corollary}{บทแทรก}[section]
%\newtheorem{problem}{โจทย์ข้อที่}

\theoremstyle{remark}
%\newtheorem*{remark}{ข้อสังเกต}
\newtheorem*{note}{สังเกต}

%\makeatletter
%\def\@maketitle{%
%	\newpage%
%	\vspace{15 in}%
%	\begin{center}%
%		{\color{black}\Huge\bfseries\@title \par}%
%		\vskip 1.5em%
%	\end{center}%
%	\par
%%	\vskip 1.5em
%	}
%\makeatother
%
\usepackage[explicit]{titlesec}
%\makeatletter
%\def\@makechapterhead#1{%
%	%%%%\vspace*{50\p@}% %%% removed!
%	{\parindent \z@ \raggedright \normalfont
%		\ifnum \c@secnumdepth >\m@ne
%		\huge\bfseries \@chapapp\space \thechapter
%		\par\nobreak
%		\vskip 20\p@
%		\fi
%		\interlinepenalty\@M
%		\Huge \bfseries #1\par\nobreak
%		\vskip 10\p@
%}}
%\def\@makeschapterhead#1{%
%	%%%%%\vspace*{50\p@}% %%% removed!
%	{\parindent \z@ \raggedright
%		\normalfont
%		\interlinepenalty\@M
%		\Huge \bfseries  #1\par\nobreak
%		\vskip 10\p@
%}}
%\makeatother
%
%\usepackage{tikz}\usetikzlibrary{shapes.misc}
%\newcommand{\titlebar}[1]{%
%	\tikz[baseline,trim left=1in,trim right=1in] {
%		\fill [yellow!35] (1in,-1.5ex) rectangle (#1 + 1.12in,3.5ex);
%	}%
%}
%\titleformat{name=\section,numberless}{\bfseries\large}{}{0in}%
%{\addcontentsline{toc}{section}{#1}%
%	\settowidth{\mylength}{#1}\titlebar{\mylength} #1}
%
%\newmdenv[%
%backgroundcolor = RoyalBlue,%
%hidealllines=true]{extr}
%

\titleformat
{\chapter}
[block]
{\bfseries\Huge}
{}
{0in}
{#1 \hfill \thechapter}
[\vspace{-2ex}%
\color{BurntOrange}\rule{\textwidth}{3pt}]

%\titleformat
%{\chapter} % command
%[display] % shape
%{\bfseries\Large\itshape} % format
%{Story No. \ \thechapter} % label
%{0.5ex} % sep
%{
%	\rule{\textwidth}{1pt}
%	\vspace{1ex}
%	\centering
%	#1
%} % before-code
%[
%\vspace{-0.5ex}%
%\rule{\textwidth}{0.3pt}
%] % after-code

%\BeforeBeginEnvironment{equation*}{\begin{singlespace}}
%	\AfterEndEnvironment{equation*}{\end{singlespace}\noindent\ignorespaces}
%\BeforeBeginEnvironment{align}{\begin{onehalfspace}}
%	\AfterEndEnvironment{align}{\end{onehalfspace}\noindent\ignorespaces}

\newcommand{\q}[1]{``#1''}

\newcommand{\fitch}[1]{
	
\begin{minipage}[l]{0.5in}%
		\begin{equation*}%
		\begin{nd}%
		#1	%
		\end{nd}%
		\end{equation*}%
		\vspace{0pt}%
\end{minipage}%

}

\makeatletter
\g@addto@macro\normalsize{%
	\setlength\abovedisplayskip{5pt}
	\setlength\belowdisplayskip{5pt}
	\setlength\abovedisplayshortskip{5pt}
	\setlength\belowdisplayshortskip{5pt}
}
\makeatother

\newcommand{\boxthis}[1]{
	\begin{textbox}%
		#1 
	\end{textbox}%
}

\DeclarePairedDelimiter{\Set}{\lbrace}{\rbrace}

